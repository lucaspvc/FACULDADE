\section{\large INTRODUÇÃO}  

Este relatório apresenta uma análise detalhada do desempenho de um sistema de comunicação baseado em filas, realizado por meio de uma simulação computacional ao longo de 100.000 segundos. O estudo é dividido em duas etapas, cada uma abordando cenários específicos que simulam tráfegos distintos: navegação web e chamadas em tempo real. O principal objetivo é avaliar o comportamento do sistema sob diferentes taxas de ocupação (60\%, 80\%, 95\% e 99\%), utilizando parâmetros matematicamente calculados para garantir que essas condições sejam precisamente atendidas.  

Ao longo da simulação, dados acumulados foram coletados em intervalos regulares de 100 segundos, permitindo uma análise detalhada das seguintes métricas de desempenho:  
\begin{itemize}  
    \item \textbf{Ocupação em função do tempo}: Monitoramento da utilização do sistema em cada cenário avaliado.  
    \item \textbf{$E[N]$ em função do tempo}: Número médio de pacotes ou clientes no sistema durante os intervalos.  
    \item \textbf{$E[W]$ em função do tempo}: Tempo médio de espera dos pacotes no sistema, determinante para a avaliação do desempenho.  
    \item \textbf{$\lambda$ em função do tempo}: Taxa de chegada de pacotes, essencial para a validação dos cálculos de ocupação.  
    \item \textbf{Erro do Teorema de Little em função do tempo}: Análise da precisão das medições acumuladas com base nos parâmetros teóricos do sistema.  
\end{itemize}  

A coleta consistente de informações permite a análise cumulativa e a comparação com os resultados teóricos previamente calculados, garantindo a validade das conclusões. Além disso, a segunda etapa do estudo explora as interações entre tráfegos de diferentes naturezas, oferecendo uma visão mais abrangente do comportamento do sistema em cenários de uso misto.