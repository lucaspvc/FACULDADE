
\section{CONCLUSÃO}

Este trabalho teve como objetivo a simulação de um roteador que recebe pacotes de navegação web e, posteriormente, a introdução de chamadas em tempo real no sistema. Na primeira etapa, focamos em modelar a chegada e o processamento dos pacotes de dados, analisando a ocupação do link e as métricas de desempenho utilizando a fórmula clássica de Little. Para isso, ajustamos o tamanho do link para diferentes níveis de ocupação e realizamos a coleta de métricas de desempenho ao longo da simulação.

Na segunda etapa, incorporamos o tráfego de chamadas em tempo real, com características distintas, e calculamos novamente as métricas de desempenho para o sistema como um todo e para o tráfego em tempo real isolado. A análise dos resultados foi feita por meio de gráficos e cálculos de Little, avaliando o impacto das mudanças no comportamento do sistema.

Os resultados demonstraram a importância da capacidade do link e das taxas de chegada dos pacotes na determinação do desempenho do sistema, permitindo uma análise detalhada do comportamento de redes sob diferentes condições de tráfego. A validação das métricas e a interpretação dos resultados são fundamentais para entender a eficiência do roteador em diferentes cenários.

Este trabalho forneceu uma boa compreensão das dinâmicas envolvidas na análise de desempenho de redes, além de evidenciar como as técnicas de simulação podem ser aplicadas para avaliar e melhorar o desempenho de sistemas de comunicação.

\section{REFERÊNCIAS BIBLIOGRÁFICAS}

\begin{itemize} 
\item Gonzaga, Flávio Barbieri. Aulas disponibilizadas pela plataforma Classroom. 
\item Site sobre Árvore Heap. \url{https://www.geeksforgeeks.org/heap-data-structure/}
 \item Emidio, Gabriele. \textit{Minicurso: Introdução ao Gnuplot}. Programa de Pós-Graduação em Física, Joinville, 2020. Disponível em: \url{https://www.udesc.br/arquivos/cct/id_cpmenu/6696/MINICURSO____Gabriele_1590606873948_6696.pdf}. 
 \end{itemize}