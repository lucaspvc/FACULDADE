\begin{center}
\begin{figure}[H]
    \centering
    \includegraphics[width=0.3\textwidth]{Imagens/logo.png}
\end{figure}

\textbf{\Large RELATÓRIO: \\ Simulação de Desempenho – Disciplina de Análise de Desempenho [DCE539] }
\end{center}

\begin{center}
\textit{
    Lucas P. O. Alves\footnote{2022.1.08.044}\\
    Professor: Flávio Barbieri Gonzaga
}
\end{center}

\begin{center}
    Universidade Federal de Alfenas
\end{center}

\begin{center}
    \textit{Neste trabalho de análise de desempenho, realizamos uma simulação computacional que avalia o comportamento de um sistema através de métricas como tempo de chegada, tempo de saída e tempo total da simulação. A partir desses dados, calculamos importantes parâmetros como $E[N]$ (número médio de clientes no sistema), $E[W]$ (tempo médio de espera no sistema), $\lambda$(taxa de chegada), ocupação e verificação do Teorema de Little. Os resultados das simulações são comparados com os valores teóricos previamente calculados.}
\end{center}
