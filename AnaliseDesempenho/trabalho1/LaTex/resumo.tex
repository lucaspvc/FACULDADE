\begin{center}
    \textbf{RESUMO}
\end{center}

O objetivo desta simulação é analisar o desempenho de um sistema em diferentes cenários de ocupação: 85\%, 90\%, 95\% e 99\%. Através de cálculos matemáticos, foram definidos parâmetros que garantem que as taxas de ocupação do sistema atendam aos valores estabelecidos. A simulação foi realizada ao longo de 100.000 segundos, e a cada 100 segundos foram feitas medições acumuladas de $E[N]$ (número médio de clientes no sistema), $E[W]$ (tempo médio de espera no sistema), $\lambda$(taxa de chegada), ocupação e verificação do Teorema de Little. Os resultados das simulações são comparados com os valores teóricos previamente calculados.

\vspace{2ex}

\textbf{Palavras-chave}: Desempenho, Gráficos, Teorema de Little, Ocupação, Número médio de clientes ($E[N]$), Tempo médio de espera ($E[W]$), Medições acumuladas, Simulação de desempenho..
