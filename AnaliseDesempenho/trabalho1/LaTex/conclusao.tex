
\section{CONCLUSÃO}
Ao longo desta análise, calculamos e discutimos o comportamento de sistemas de filas em diferentes condições de ocupação (85\%, 90\%, 95\% e 99\%). A partir dos cálculos de $\rho$, $T_{\text{chegada}}$, e $T_{\text{atendimento}}$, foi possível observar como o sistema se comporta em termos de taxa de ocupação, número médio de elementos no sistema ($E[n]$), tempo médio de espera ($E[w]$) e taxa de chegada ($\lambda$) ao longo do tempo.

Os resultados mostraram que, conforme a ocupação do sistema aumenta, o número de elementos e o tempo de espera também crescem, especialmente em cenários com ocupação de 99\%, onde o sistema atinge um ponto de saturação. Gráficos como os de $E[n]$ e $E[w]$ indicam que, em sistemas com maior carga, os tempos de espera e o acúmulo de entidades aumentam significativamente. A análise do erro de Little reforçou que, em altos níveis de ocupação, a Lei de Little se torna menos precisa, apresentando maior volatilidade nos resultados.

Portanto, a simulação confirma a importância de gerenciar a ocupação do sistema para evitar saturações e garantir o equilíbrio entre a taxa de atendimento e a taxa de chegada, assegurando assim um desempenho adequado do sistema de filas.

\section{REFERÊNCIAS BIBLIOGRÁFICAS}

\begin{itemize} 
\item Gonzaga, Flávio Barbieri. Aulas disponibilizadas pela plataforma Classroom. 
 \item Emidio, Gabriele. \textit{Minicurso: Introdução ao Gnuplot}. Programa de Pós-Graduação em Física, Joinville, 2020. Disponível em: \url{https://www.udesc.br/arquivos/cct/id_cpmenu/6696/MINICURSO____Gabriele_1590606873948_6696.pdf}. 
 \end{itemize}